\documentclass[a4paper,10pt]{article}
\usepackage[spanish,es-tabla]{babel}
\usepackage[utf8]{inputenc}

%opening

\begin{document}

\begin{flushleft}
\begin{figure}
 \includegraphics[scale=0.4]{logo}
\end{figure}
 Universidad Simón Bolívar\\
 Inteligencia Artificial II\\
 Grupo 01 \\
 Bernardo Morales, Leonardo Ramos y Rubén Serradas\\
\end{flushleft}

\begin{abstract}

\end{abstract}

\section{Descripción de la implementación}

\indent Se utilizó el lenguaje de programación \verb|Python| para la realización del 
clasificador de las tres clases de la flor Iris (Setosa, Versicolor y Virginica), 
partiendo del sistema GABIL realizado por Kenneth A. De Jong et al. (\emph{Using Ge-
netic Algorithms for Concept Learning}). Aprovechamos la librería DEAP (\emph{Distributed Evolutionary Algorithms in Python})
para poder concentrarnos en conceptos fundamentales del algoritmo genético realizado. Sin embargo, se realizó
la función de Fitness y Crossover pues la librería no contaba con estas para un sistema GABIL.\\
\indent Algo que es bastante relevante para el desarrollo de un sistema GABIL es la transformación de 
los datos, se analizaron cada uno de estos con un script en el lenguaje de programación R
realizando un histograma de frecuencias y dividendo los datos de la manera expresada
por las gráficas mostradas por el script.\\
\indent Para utilizar el clasificador tan solo es necesario escribir \verb|python clasificador_iris.py| y 
el programa imprimirá un mensaje de ayuda para su uso.

\section{Descripción del Algoritmo Genético}

\section{Descripción de Experimentos Realizados}

\section{Discusión}



\end{document}
