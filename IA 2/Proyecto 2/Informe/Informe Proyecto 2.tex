\documentclass[a4paper,10pt]{article}
\usepackage[spanish,es-tabla]{babel}
\usepackage[utf8]{inputenc}
\usepackage{amsmath}
\usepackage{graphicx}

%opening

\begin{document}

\begin{flushleft}
\begin{figure}
 \includegraphics[scale=0.4]{logo}
\end{figure}
 Universidad Simón Bolívar\\
 Inteligencia Artificial II\\
 Grupo 01 \\
 Bernardo Morales, Leonardo Ramos y Rubén Serradas\\
\end{flushleft}

\begin{abstract}

\indent Se realizó un clasificador para las tres clases de flores de Iris del conjunto de datos de Iris Setosa
utilizando un algoritmo genético. Se partió  del sistema GABIL realizado por Kenneth A. De Jong et al.\\
\indent Se llegó hasta una precisión de hasta 92 \% con los datos de prueba. La descripción de la implementación,
mejor algoritmo genético obtenido, experimentos realizados y discusión se encuentra a continuación.

\end{abstract}

\section{Descripción de la implementación}

\indent Se utilizó el lenguaje de programación \verb|Python| para la realización del 
clasificador de las tres clases de la flor Iris (Setosa, Versicolor y Virginica), 
partiendo del sistema GABIL realizado por Kenneth A. De Jong et al. (\emph{Using Ge-
netic Algorithms for Concept Learning}). Aprovechamos la librería DEAP (\emph{Distributed Evolutionary Algorithms in Python})
para poder concentrarnos en conceptos fundamentales del algoritmo genético realizado. Sin embargo, se realizó
la función de Fitness y Crossover pues la librería no contaba con estas para un sistema GABIL.\\
\indent Algo que es bastante relevante para el desarrollo de un sistema GABIL es la transformación de 
los datos, se analizaron cada uno de estos con un script en el lenguaje de programación R
realizando un histograma de frecuencias y dividendo los datos de la manera expresada
por las gráficas mostradas por el script.\\
\indent Para utilizar el clasificador tan solo es necesario escribir \verb|python| \\ \verb|clasificador_iris.py| y 
el programa imprimirá un mensaje de ayuda para su uso.

\section{Descripción del Algoritmo Genético}

\indent Del conjunto de experimentos realizados se observó que la mejor manera de elegir
padres para nuestro caso de estudio es elegir los mejores para su reproducción y una vez
obtenido los hijos de estos la mejor manera de elegir a los sobrevivientes es mediante el
método del elitismo. Esto probablemente sea debido a que el conjunto de prueba no era 
muy grande y no era tan importante mantener una gran diversidad en nuestro caso de estudio.\\
\indent El programa \verb|clasificador_iris.py| nos permite elegir entre varias opciones 
para la elección de padres (ruleta y mejor) y sobrevivientes (elitismo o ruleta).\\
\indent La tasa de mutación usada por la mejor configuración fue de 0.02 y la de 
crossover 0.1. Nuestra función de fitness era la misma que la del sistema GABIL presentada
en el artículo de Kenneth A. De Jong et al., $fitness(individuo) = (porcentajeCorrecto)^2$ , 
con la particularidad de que podíamos penalizar una longitud determinada de individuo. La mejor
configuración no usaba la opción de la penalización.

\section{Descripción de Experimentos Realizados}

\indent Se hicieron 100 ejecuciones para cada una de las configuraciones usadas tomando
el promedio del resultado, los resultados junto con las configuraciones 
usadas se encuentras en la tabla \ref{tabla}. Para el entrenamiento del algoritmo
se usó un conjunto de 12 ejemplos bien representativos, 4 por cada tipo de Iris, esto
demuestra la gran capacidad que tiene un sistema GABIL para la clasificación pues
los resultados de la mejor configuración se acercan a los obtenidos en el proyecto
anterior por una red neuronal.

\begin{table}[!hbtp]
\begin{tabular}{|l|p{2.4cm}|p{2.4cm}|p{2.4cm}|}
\hline
Resultados & Tasa de Mutación & Tasa de Crossover & Comienzo Penalización \\
\hline
92.02898550724623 & 0.02 & 0.1 & Ninguna \\
\hline
86.95652173913025 & 0.02 & 0.1 & 10 \\
\hline
80.43478260869578 & 0.02 & 0.1 & 20 \\
\hline
90.57971014492776 & 0.02 & 0.4 & Ninguna \\
\hline
86.95652173913025 & 0.02 & 0.4 & 10 \\
\hline
77.53623188405798 & 0.02 & 0.4 & 20 \\
\hline
78.26086956521726 & 0.02 & 0.8 & Ninguna \\
\hline
67.39130434782602 & 0.02 & 0.8 & 10 \\
\hline
66.66666666666674 & 0.02 & 0.8 & 20 \\
\hline
23.920289855072486 & 0.05 & 0.1 & Ninguna \\
\hline
86.95652173913025 & 0.05 & 0.1 & 10 \\
\hline
22.2753623188406 & 0.05 & 0.1 & 20 \\
\hline
91.30434782608688 & 0.05 & 0.4 & Ninguna \\
\hline
78.98550724637693 & 0.05 & 0.4 & 10 \\
\hline
84.0579710144929 & 0.05 & 0.4 & 20 \\
\hline
82.60869565217398 & 0.05 & 0.8 & Ninguna \\
\hline
84.78260869565207 & 0.05 & 0.8 & 10 \\
\hline
79.71014492753608 & 0.05 & 0.8 & 20 \\
\hline
66.66666666666674 & 0.1 & 0.1 & Ninguna \\
\hline
76.08695652173905 & 0.1 & 0.1 & 10 \\
\hline
90.57971014492776 & 0.1 & 0.1 & 20 \\
\hline
60.86956521739126 & 0.1 & 0.4 & Ninguna \\
\hline
81.15942028985502 & 0.1 & 0.4 & 10 \\
\hline
86.95652173913025 & 0.1 & 0.4 & 20 \\
\hline
78.26086956521726 & 0.1 & 0.8 & Ninguna \\
\hline
85.50724637681176 & 0.1 & 0.8 & 10 \\
\hline
86.95652173913025 & 0.1 & 0.8 & 20 \\
\hline
\end{tabular}
\caption{Tabla de resultados para el sistema GABIL con distintas configuraciones usando los mejores 
padres y elitismo com``o método de selección de sobrevivientes. \label{tabla}}
\end{table}

\section{Discusión}

\indent Observando la tabla \ref{tabla} la mejor configuración obtenida para nuestro caso de estudio 
fue la de selección de los mejores padres con elitismo para método de selección de sobrevivientes con 
tasa de mutación 0.02, crossover 0.1 y función fitness sin ningún tipo de penalización.\\
\indent La función fitness usada, como se explicó anteriormente, es la misma que la del articulo 
del sistema GABIL. Esta compara  para cada ejemplo, si alguna de las reglas del individuo tiene los mismas características 
del ejemplo con el resultado esperado se le suma un punto. Al final se saca el promedio de los aciertos dividido entre
el total y este resultado se eleva al cuadrado. Si se esta considerando la penalización se observa que la regla 
no sea mayor que una longitud deseada, si esta se pasa entonces el fitness de la regla muy larga es 0. 
Esta penalización puede ser útil para clasificadores muy grandes que tengan muchos ejemplos a considerar y 
características ya que se podría tener un ``overfitting'' con las reglas muy largas y el clasificador
se convertiría en una especie de árbol de decisiones.\\ 
\indent Por último, el mejor conjunto de reglas hallado por el algoritmo genético se describe a continuación:

\begin{itemize}
 \item Sin importar el largo del sépalo, con anchura 2-2.5 o 3-4 cm. Pétalo de largo 1-3 cm y ancho 0.5 a 1.5 o 2-2.5 puede ser Iris Setosa o Versicolor.   
 \item Con largo de sépalo 7-8 cm., anchura 2-2.5 o 3-4 cm. Largo del pétalo 1-2 o 3-4  cm. y ancho 0.0 a 1.0 o 1.5-2.0 o 2.5 en adelante 
 tenemos Iris Virginica o Versicolor.
 \item Con largo de sépalo 6-7 cm., anchura de 2-3 o 3.5 cm. en adelante. Largo del pétalo 2-3 o 6-7 cm. y ancho 0.0-0.5 o 1.0-1.5 o 2.0-2.5 cm. 
 puede ser cualquier clase.
 \item Con largo de sépalo 5-6 o 7-8 cm. con anchura de 2-2.5 o 3-4 cm. Pétalo de largo 1-3 o 4-5 cm. con ancho 0.5-1.5 o 2-2.5 puede ser cualquier clase. 
 \item Con largo de sépalo 4-5 o 6-8 cm. ancho de 2.5-3.5 o 4.0-4.5 cm. Largo de pétalo 4-6 y ancho 0.0-0.5 o 1.0-1 o 2.0-3.0 puede ser cualquier clase.
 \item Largo de sépalo 5-7 cm., ancho 2.0-3.0 o 3.5-4.0 cm. Largo de pétalo 3-7 cm., ancho de 1.5-3.0 cm. puede ser cualquier clase.
 \item Largo de sépalo 4-8 cm. ancho de  2.0-2.5 o 3.0-3.5 cm. Largo de pétalo 2-6 cm. y ancho 0.0-0.5 o 1.0-3.0 cm. puede ser cualquier clase.
 \item Largo de sépalo 4-8 cm., ancho de 2.5-3.0 cm. Largo del pétalo 1-3, 4-5 o 6-7  cm. y ancho de 0.5-2.5 cm. puede ser de cualquier clase.
 \item Largo del sépalo 5-7 cm., ancho de 2.0-2.5 o 4.0-4.5 cm., largo del pétalo 1-2 o 3-4 o 5-6 cm. y 0.0-3.0 de ancho puede ser cualquier clase.
 \item Largo del sépalo 4-6 o 7-8 cm., ancho de 3.5-4.5 cm. Largo del pétalo 2-3 o 4-5 o 6-7 cm., ancho de 0.0-3.0 cm. puede ser Iris Setosa o Virginica.
 \item Largo del sépalo 4-5 o 6-7 cm., ancho de 3.0-4.0 cm. Largo del pétalo 1-2 o 3-4 o 5-7 cm., ancho de 0.0-3.0 cm. puede ser Iris Setosa o Versicolor.
 \item Largo del sépalo 5-6 cm., ancho de 2.5-3.5 o 4.0-4.5 cm. Largo del pétalo 2-3 o 4-7 con ancho de 0.0-3.0 cm. puede ser Iris Virginica o Versicolor.
 \item Largo del sépalo 4-5 cm., ancho de 2.0-3.0 o 3.5-4.0 cm. Largo del pétalo 1-2 o 3-7 cm. y ancho de 0.0-2.5 cm. puede ser Iris Setosa o Virginica
 \item Largo del sépalo 7-8 cm. con ancho de  2.0-2.5 o 3.0-3.5 o 4.0-4.5 cm. Largo del pétalo 2-7 cm. y ancho de 0.0-3.0 cm. puede ser Iris Setosa o Versicolor
 \item Largo del sépalo 6-8 cm. con ancho de 2.5-3.0 o 3.5-4.0 cm. Largo del pétalo 1-7 cm. y ancho de 0.0-1.5 o 2.0-3-0 cm. es una Iris Virginica.
 \item Largo del sépalo 4-8 cm. con ancho de 2.5-3.0 o 3.5-4.5 cm. Largo del pétalo 1-7 cm. y ancho de este de 1.0-2.0 o 2.5-3.0 cm. no pertenece a ninguna clase de Iris.
 \item Largo del sépalo 4-5 o 6-7 cm., ancho de 2.0-2.5 o 3.0-4.5 cm. Largo del pétalo 1-7 cm., ancho de este de 0.5-2.5 cm. es Iris Versicolor.
\end{itemize}


\end{document}
